%%
%% This is file `tikzposter-template.tex',
%% generated with the docstrip utility.
%%
%% The original source files were:
%%
%% tikzposter.dtx  (with options: `tikzposter-template.tex')
%%
%% This is a generated file.
%%
%% Copyright (C) 2014 by Pascal Richter, Elena Botoeva, Richard Barnard, and Dirk Surmann
%%
%% This file may be distributed and/or modified under the
%% conditions of the LaTeX Project Public License, either
%% version 2.0 of this license or (at your option) any later
%% version. The latest version of this license is in:
%%
%% http://www.latex-project.org/lppl.txt
%%
%% and version 2.0 or later is part of all distributions of
%% LaTeX version 2013/12/01 or later.
%%


\documentclass{tikzposter} %Options for format can be included here

\usepackage{todonotes}

\usepackage[tikz]{bclogo}
\usepackage{lipsum}
\usepackage{amsmath}

\usepackage{booktabs}
\usepackage{longtable}
\usepackage[absolute]{textpos}
\usepackage[it]{subfigure}
\usepackage{graphicx}
\usepackage{cmbright}
%\usepackage[default]{cantarell}
%\usepackage{avant}
%\usepackage[math]{iwona}
\usepackage[math]{kurier}
\usepackage[T1]{fontenc}


%% add your packages here
\usepackage{hyperref}
% for random text
\usepackage{lipsum}
\usepackage[english]{babel}
\usepackage[pangram]{blindtext}

\colorlet{backgroundcolor}{blue!10}

 % Title, Author, Institute
\title{Bike Sharing Demand Prediction}
\author{Dong Zhu}
\institute{Deakin University, Australia
}
%\titlegraphic{logos/tulip-logo.eps}

%Choose Layout
\usetheme{Wave}

%\definebackgroundstyle{samplebackgroundstyle}{
%\draw[inner sep=0pt, line width=0pt, color=red, fill=backgroundcolor!30!black]
%(bottomleft) rectangle (topright);
%}
%
%\colorlet{backgroundcolor}{blue!10}

\begin{document}


\colorlet{blocktitlebgcolor}{blue!23}

 % Title block with title, author, logo, etc.
\maketitle

\begin{columns}
 % FIRST column
\column{0.5}% Width set relative to text width

%%%%%%%%%% -------------------------------------------------------------------- %%%%%%%%%%
 %\block{Main Objectives}{
%  	      	\begin{enumerate}
%  	      	\item Formalise research problem by extending \emph{outlying aspects mining}
%  	      	\item Proposed \emph{GOAM} algorithm is to solve research problem
%  	      	\item Utilise pruning strategies to reduce time complexity
%  	      	\end{enumerate}
%%  	      \end{minipage}
%}
%%%%%%%%%% -------------------------------------------------------------------- %%%%%%%%%%


%%%%%%%%%% -------------------------------------------------------------------- %%%%%%%%%%
\block{Introduction}{
  Bike sharing systems are a means of renting bicycles where the process
   of obtaining membership, rental, and bike return is automated via 
   a network of kiosk locations throughout a city. Using these systems, 
   people are able rent a bike from a one location and return it to a 
   different place on an as-needed basis. Currently, there are over 500 bike-sharing programs around the world.

  The data generated by these systems makes them attractive for researchers 
  because the duration of travel, departure location, arrival location, 
  and time elapsed is explicitly recorded. Bike sharing systems therefore 
  function as a sensor network, which can be used for studying mobility in a city. 
  In this competition, participants are asked to combine historical usage patterns 
  with weather data in order to forecast bike rental demand in the Capital Bikeshare 
  program in Washington, D.C.
    
}
%%%%%%%%%% -------------------------------------------------------------------- %%%%%%%%%%


%%%%%%%%%% -------------------------------------------------------------------- %%%%%%%%%%
\block{Dataset Description}{
\begin{itemize}
    \item
    %\emph{Group Outlying Aspects Mining}
    The competition provides hourly rental data spanning two years. For this competition, the training set is comprised of the first 19 days of each month, while the test set is the 20th to the end of the month. You must predict the total count of bikes rented during each hour covered by the test set, using only information available prior to the rental period.

    \item
    Data Fields
    \end{itemize} 
      
  
      \begin{tabular}{ c | c }
        \toprule
        Attribute     &  Description          \\
        \midrule
        datetime       &  hourly date + timestamp  \\
        season      & 1 = spring, 2 = summer, 3 = fall, 4 = winter\\ 
        holiday  & whether the day is considered a holiday\\
        workingday & whether the day is neither a weekend nor holiday\\
        weather    & 1: Clear,2: Mist, 3: Light Snow, Light Rain , 4: Extreme weather\\ 
        temp  & temperature in Celsius\\
        atemp  & "feels like" temperature in Celsius\\
        humidity  & relative humidity\\
        windspeed  &  wind speed\\
        casual  & number of non-registered user rentals initiated\\
        registered  & number of registered user rentals initiated\\
        count  & number of total rentals \\
    
        \bottomrule
      \end{tabular}



}
%%%%%%%%%% -------------------------------------------------------------------- %%%%%%%%%%


%%%%%%%%%% -------------------------------------------------------------------- %%%%%%%%%%

%\note{Note with default behavior}

%\note[targetoffsetx=12cm, targetoffsety=-1cm, angle=20, rotate=25]
%{Note \\ offset and rotated}

 % First column - second block


%%%%%%%%%% -------------------------------------------------------------------- %%%%%%%%%%
\block{Data Visualization}{
  	We propose the \emph{GOAM} algorithm to solve the research problem of
    \emph{Group Outlying Aspects Mining}.
  	The \emph{GOAM} algorithm includes three major steps.
%    1) Group Feature Extraction,
%    2) Outlying Degree Scoring, and
%    3) Outlying Aspects Identification.
  	
\begin{tikzfigure}%[Overall architecture of \emph{GOAM} algorithm]
%  \includegraphics[width=0.8\linewidth]{figures//framework.pdf}
    \missingfigure[figcolor=white]{Testing figcolor}
\end{tikzfigure}
		
\begin{description}
  	\item[Group Feature Extraction]
  	Let $f_1$, $f_2$, $f_3$ represent three features of $G_q$.
    We count the frequency of each value for one feature.
    Then use the histogram to represent each feature.
    Similarly,
    we can extract other features for each group.

%    \item
%    The histogram of $G_q$ on three features are as follows.
\end{description}

\begin{center}
    \begin{minipage}{0.3\linewidth}
    \centering
    \begin{tikzfigure}
    \missingfigure[figcolor=white]{Testing figcolor}
    {\small{Histogram of $G_q$ on $f_1$}}
    \end{tikzfigure}%
    \end{minipage}
    \hfill
    \begin{minipage}{0.3\linewidth}
    \centering
    \begin{tikzfigure}
    \missingfigure[figcolor=white]{Testing figcolor}
    {\small{Histogram of $G_q$ on $f_2$}}
    \end{tikzfigure}%
    \end{minipage}
    \hfill
    \begin{minipage}{0.3\linewidth}
    \centering
    \begin{tikzfigure}
    \missingfigure[figcolor=white]{Testing figcolor}
    {\small{Histogram of $G_q$ on $f_3$}}
    \end{tikzfigure}%
    \end{minipage}
\end{center}
\begin{description}
\item[Outlying Degree Scoring]
    In this step,
    we first calculate the \emph{earth mover distance} (EMD) of one feature among different groups.
    The earth mover distance reflects the minimum mean distance
    between groups on one feature.
    So,
    we utilize the EMD to measure the difference between groups of each feature.
\end{description}
}
%%%%%%%%%% -------------------------------------------------------------------- %%%%%%%%%%


% SECOND column
\column{0.5}
 %Second column with first block's top edge aligned with with previous column's top.

%%%%%%%%%% -------------------------------------------------------------------- %%%%%%%%%%
\block{GOAM Algorithm}{
\begin{description}
    \item
    Second,
    based on the \emph{earth move distance},
    we calculate the outlying degree.
\end{description}

\begin{tikzfigure}%[Overall architecture of \emph{GOAM} algorithm]
    \missingfigure[figcolor=white]{Testing figcolor}
\end{tikzfigure}
  where $G_q$ is the query group,
  $n$ is the number of compare groups,
  and $h_{k_s}$ is the histogram representation of $G_k$ in the subspace $s$.

\begin{description}
  	\item[Outlying Aspects Identification]
    In this step,
    based on the value of outlying degree
    we will identify the group outlying aspects.
    If a feature's outlying degree is greater than a threshold,
    the more likely the feature is group outlying aspect.
    When the dimensionality of features is high,
    we adopt a stage-wise candidate subspace construction strategy to
    alleviate the exponential explosion.
\end{description}
}
%%%%%%%%%% -------------------------------------------------------------------- %%%%%%%%%%
% Second column - first block


%%%%%%%%%% -------------------------------------------------------------------- %%%%%%%%%%
\block[titleleft]{Experiment}
{
\begin{description}
  	\item[Synthetic Dataset] contains $10$ groups and $8$ features.
    Each group consists of $10$ members,
    and each member has $8$ features.
\end{description}
\vspace{.5cm}
\begin{tabular}{ c | c | c | c }
    \toprule
    Method     &  Truth Outlying Aspects    & Identified Aspects & Accuracy      \\
    \midrule
    GOAM       &  $\{F_1\}$, $\{F_2F_4\}$   &  $\{F_1\}$, $\{F_2F_4\}$    & 100\%    \\

     Arithmetic Mean based OAM &  $\{F_1\}$, $\{F_2F_4\}$   &  $\{F_4\}$, $\{F_2\}$    &  0\% \\

     Median based OAM &  $\{F_1\}$, $\{F_2F_4\}$   &  $\{F_2\}$, $\{F_4\}$    &           0\% \\
     \bottomrule
\end{tabular}
\vspace{.2cm}
\begin{description}
    \item
    It can be observed that the GOAM method can identify the trivial outlying features
    and non-trivial outlying subspaces correctly and is obvious from the table
    that the accuracy of GOAM is the best, which is ($100\%$).
\end{description}

\begin{description}
\item[NBA Dataset] was collected from Yahoo Sports
website (\url{http://sports.yahoo.com.cn/nba}).
The data include all teams from the six divisions,
and each player in the team has $12$ features.
\end{description}
\vspace{.5cm}
\begin{tabular}{ c | c | c }
    \toprule
    Teams                   & Trivial Outlying Aspects  & NonTrivial Outlying Aspects    \\
    \toprule
    Cleveland Cavaliers     & \{3FA\}                   & \{FGA, FT\%\}, \{FGA, FG\%\} \\
    Orlando Magic           & \{Stl\}                   & None                         \\
    Milwaukee Bucks         & \{To\}, \{FTA\}           & \{FGA, FTA\}, \{3FA, FTA\}     \\
%    Golden State Warriors   & \{FG\%\}                  & \{FT\%, Blk\}, \{FGA, 3PT\%, FTA\}\\
%    Utah Jazz               & \{Blk\}                   & \{3FA, 3PT\%\}                    \\
    New Orleans Pelicans    & \{FT\%\}, \{FTA\}         & \{FTA, Stl\}, \{FTA, To\}          \\
    \bottomrule
\end{tabular}
           
\begin{minipage}{0.5\linewidth}
    \centering
    \begin{tikzfigure}
    \missingfigure[figcolor=white]{Testing figcolor}

    {\small{New Orleans Pelicans on FT\%}}
    \end{tikzfigure}%
\end{minipage}
\hfill
\begin{minipage}{0.5\linewidth}
    \centering
    \begin{tikzfigure}
    \missingfigure[figcolor=white]{Testing figcolor}

    {\small{New Orleans Pelicans on FTA}}
    \end{tikzfigure}%
\end{minipage}
\vspace{.2cm}
\begin{description}
\item
\texttt{New Orleans Pelicans} has more players with
lower \{free throw percentage\}, \{free throws attempted\}.
\end{description}
}
%%%%%%%%%% -------------------------------------------------------------------- %%%%%%%%%%


% Second column - second block
%%%%%%%%%% -------------------------------------------------------------------- %%%%%%%%%%
\block[titlewidthscale=1, bodywidthscale=1]
{Conclusion}
{
\begin{description}
  \item[Problem Definition]
  Formalize the problem of Group Outlying Aspects Mining by extending outlying aspects mining.

  \item[GOAM algorithm]
  Propose GOAM algorithm to solve the \emph{Group}\\
  \emph{Outlying Aspects Mining} problem.

  \item[Strategies]
  Utilize the pruning strategies to \\ reduce time complexity.
\end{description}
}
%%%%%%%%%% -------------------------------------------------------------------- %%%%%%%%%%


% Bottomblock
%%%%%%%%%% -------------------------------------------------------------------- %%%%%%%%%%
\colorlet{notebgcolor}{blue!20}
\colorlet{notefrcolor}{blue!20}
\note[targetoffsetx=8cm, targetoffsety=-4cm, angle=30, rotate=15,
radius=2cm, width=.26\textwidth]{
Acknowledgement
\begin{itemize}
    \item
    International Cooperation Project (Y7Z0511101)
    of IIE,
    Chinese Academy of Sciences
 \end{itemize}
}

%\note[targetoffsetx=8cm, targetoffsety=-10cm,rotate=0,angle=180,radius=8cm,width=.46\textwidth,innersep=.1cm]{
%Acknowledgement
%}

%\block[titlewidthscale=0.9, bodywidthscale=0.9]
%{Acknowledgement}{
%}
%%%%%%%%%% -------------------------------------------------------------------- %%%%%%%%%%

\end{columns}


%%%%%%%%%% -------------------------------------------------------------------- %%%%%%%%%%
%[titleleft, titleoffsetx=2em, titleoffsety=1em, bodyoffsetx=2em,%
%roundedcorners=10, linewidth=0mm, titlewidthscale=0.7,%
%bodywidthscale=0.9, titlecenter]

%\colorlet{noteframecolor}{blue!20}
\colorlet{notebgcolor}{blue!20}
\colorlet{notefrcolor}{blue!20}
\note[targetoffsetx=-13cm, targetoffsety=-12cm,rotate=0,angle=180,radius=8cm,width=.96\textwidth,innersep=.4cm]
{
\begin{minipage}{0.3\linewidth}
\centering
\includegraphics[width=24cm]{logos/tulip-wordmark.eps}
\end{minipage}
\begin{minipage}{0.7\linewidth}
{ \centering
 The $11^{th}$ International Conference on Knowledge Science,
  Engineering and Management (KSEM 2018),
  17-19/08/2018, Changchun, China
}
\end{minipage}
}
%%%%%%%%%% -------------------------------------------------------------------- %%%%%%%%%%


\end{document}

%\endinput
%%
%% End of file `tikzposter-template.tex'.
